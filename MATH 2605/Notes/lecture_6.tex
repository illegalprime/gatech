\documentclass{article}
\usepackage{amsmath}
\title{Lecture 6}
\begin{document}

\maketitle

\subsection{Review of Last Friday}

When you row reduce a matrix you get pivot and non-pivot columns.
Pivot columns are those whose bottom halfs follow the identity matrix pattern.

$$
\text{Nullity}(A) = \text{Number of non-pivot columns}
$$

$$
\text{Rank}(A) = \text{Number of pivot columns}
$$

$$
\text{Rank}(A) + \text{Nullity}(A) = \text{Total number of columns}
$$

\section{Inverse Matrices}

Given a matrix $A$, the inverse of $A$ is a matrix $B$ such that (where $I$ is the identity matrix):
$$
A*B = I = B*A
$$

For a $2 \times 2$ matrix we can find the inverse like:
$$
\begin{bmatrix}
    a & b \\
    c & d
\end{bmatrix}^{-1} = \frac{1}{ad - bc} \begin{bmatrix}
    d & -b \\
    -c & a \\
\end{bmatrix}
$$

For bigger matrices the process becomes harder and its better to not have an equation but an algorithm.
To find the inverse of an $n \times n$ matrix augment $A$ with $I$ and perform a row reduction:
$$
\lbrack A \mid I \rbrack \sim \lbrack I \mid B \rbrack \to B = A^{-1}
$$

\section{Basis \& Linear Mapping}

\subsection{Coordinates in a non-standard Basis}
Let a basis $S$ be $\{ \vec{v_1}, \ldots, \vec{v_n} \}$. Any vector $\vec{x}$ in $S$ is written uniquely as (where $c_k \in R$):
$$
\vec{x} = c_1 \vec{v_1} + \cdots + c_n \vec{v_n}
$$

The coordinates of $\vec{x}$ in the basis of $\vec{v}$ is written in the following way:
$$
\bar{\psi}_{\{\vec{v}\}}(\vec{x}) = \begin{pmatrix}
    c_1 \\
    c_2 \\
    \vdots \\
    c_n
\end{pmatrix}
$$

\paragraph{Example} $\{3, 2x-1, 2x^2 + x, x^3 - 2x^2 + 1\}$ is a basis of polynomial space of order 3. Find the coordinates of $3 + 2x^2 + x^3$ in this basis. \\

Doing this is equivalent to solving this equation:
$$
3 + 2x^2 + x^3 = c_1(3) + c_2(2x-1) + c_3(2x^2 + x) + c_4(x^3 - 2x^2 + 1)
$$

What are the coefficients of $x^3$?
$$
1x^3 = c_4 x^3 \to c_4 = 1
$$

What about $x^2$?
$$
2x^2 = 2 c_3 x^2 - 2 x^2 \to c_3 = 2
$$

No $x$ term, so:
$$
0x = 2 c_2 x + 2x \to c_2 = -1
$$

What about 3?
$$
3 = 3 c_1 + 1 + 1 \to c_1 = \frac{1}{3}
$$

therefore the coordinates are: $(1, 2, -1, \frac{1}{3})$.

\section{Linear Mapping}
$T: V \to W$ is a linear mapping if $T(\vec{0}) = \vec{0}$ and $T(\alpha \vec{v_1} + \beta \vec{v_2}) = \alpha T(\vec{v_1}) + \beta T(\vec{v_2})$

A linear mapping between $V = \{1, x, x^2\}$ and $W = \{1, x\}$ can be seen as the derivative from an order 2 to an order 1 polynomial. \\

This will be explained further next lecture $\ldots$

\end{document}
