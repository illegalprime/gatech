\documentclass{article}
\usepackage{amsmath}
\usepackage{amssymb}

\title{Lecture 3}
\begin{document}

\maketitle

\section{Review of Linear Algebra}

\paragraph{vector in $\mathbb{R}^3$:}
$$
\vec{x} = \begin{pmatrix}
            x_1 \\
            x_2 \\
            \vdots \\
            x_n
\end{pmatrix}
$$

\paragraph{Transpose of a vector:}
$$
\vec{x}^t = (x_1, x_2, \ldots, x_n)
$$

\paragraph{Linear combination (span):}
$$
\vec{x} = c_1 x_1 + c_2 x_2 + \cdots + c_n x_n
$$

\paragraph{Linear independence:}
Vectors $\vec{x_1}, \vec{x_2}, \ldots, \vec{x_k}$ are linearly independent if
$$
c_1 \vec{x_1} + c_2 \vec{x_2} + \cdots + c_k \vec{x_k} = \vec{0}
$$
Which implies $c_1 = c_2 = \cdots = c_k = 0$

\paragraph{Dot product as matrices:}
For a vector $\vec{x} \in \mathbb{R}^n$ and $\vec{y} \in \mathbb{R}^n$ their dot product is:
$$
\vec{x} \cdot \vec{y} = \langle x_1, \ldots, x_n \rangle \left \langle \begin{matrix}
    y_1, \\
    \vdots, \\
    y_n
\end{matrix} \right \rangle = \vec{x}^T \vec{y}
$$

\paragraph{Solving a system of equations using a matrix}
For a given system of equations:
\begin{align*}
    a - 2b + c &= 0 \\
    2b - 8c &= 8 \\
    -4a + 5b + 9c &= -9
\end{align*}

The matrix form of the above system would be:
$$
\begin{bmatrix}
    1 & -2 & 1 \\
    0 & 2 & -8 \\
    -4 & 5 & 9
\end{bmatrix}
\begin{pmatrix}
    a \\
    b \\
    c
\end{pmatrix}
=
\begin{pmatrix}
    0 \\
    8 \\
    -9
\end{pmatrix}
$$

Let $A$ be the $3 \times 3$ matrix and $\vec{c}$ be the vector that $A$ is multiplied by.

If $A\vec{c} = \vec{0}$ then this system has a unique solution, which is just $\vec{c} = \vec{0}$

\paragraph{The null space} of matrix $A$
$$
N(A) = \{ \vec{x} \in \mathbb{R}^k \text{ such that } A\vec{x} = \vec{0} \}
$$

\paragraph{The column space} of matrix $A$ is
$$
R(A) = Col(A) = \{ \vec{y} \in \mathbb{R}^n \text{ where } \vec{y} = A\vec{x} \text{ for some } \vec{x} \in \mathbb{R}^x \}
$$

\paragraph{The dimension of $N(A)$} is called the nullity of $A$.
\paragraph{The dimension of $R(A)$} is called the rank of $A$.

\paragraph{The rank-nullity theorem}
$$
\text{Rank}(A) + \text{Nullity}(A) = K = \text{columns in } A
$$

\paragraph{The basis} of a $S$ where $S$ is a subspace of $\mathbb{R}^n$ is $\vec{v_1}, \ldots, \vec{v_m}$ if $S$ spans $\{ \vec{v_1}, \ldots, \vec{v_m} \}$ and $\vec{v_1}, \ldots, \vec{v_m}$ are linearly independent and $m \leq n$. Note that the dimension of $S$ is $m$.

As an example, $\langle 0, 1 \rangle$ and $\langle 1, 0 \rangle$ consist of the basis of $\mathbb{R}^2$. Note that the vectors that make up a basis for any space are not unique.

\subsection{Example}
Find the nullspace, nullity, columnspace, and rank of $A$.
$$
A = \begin{bmatrix}
    1 & 2 & -1 & 3 & 0 \\
    -1 & -2 & 2 & -2 & -1 \\
    1 & 2 & 0 & 4 & 0 \\
    0 & 0 & 2 & 2 & -1
\end{bmatrix}
$$

\paragraph{To the nullspace} we need to find $\vec{x}$ so that $A\vec{x} = \vec{0}$. We need to row reduce $A$ augmented with a zero vector (the same as solving $A\vec{x} = \vec{0}$).
$$
\begin{bmatrix}
    1 & 2 & -1 & 3 & 0 & 0\\
    -1 & -2 & 2 & -2 & -1 & 0\\
    1 & 2 & 0 & 4 & 0 & 0\\
    0 & 0 & 2 & 2 & -1 & 0
\end{bmatrix}
$$

\end{document}
